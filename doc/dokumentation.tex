\documentclass[
a4paper,
oneside,
10pt,
fleqn,
headsepline,
toc=listofnumbered, 
bibliography=totocnumbered]{scrartcl}

% deutsche Trennmuster etc.
\usepackage[T1]{fontenc}
\usepackage[utf8]{inputenc}
\usepackage[english, ngerman]{babel} % \selectlanguage{english} if  needed
\usepackage{lmodern} % use modern latin fonts

% Custom commands
\newcommand{\AUTHOR}{Michael Wieland}
\newcommand{\SECONDAUTHOR}{Fabian Hauser}
\newcommand{\INSTITUTE}{Hochschule für Technik Rapperswil}
\newcommand{\GITHUB}{https://github.com/michiwieland/hsr-zusammenfassungen}
\newcommand{\LICENSEURL}{https://www.gnu.org/licenses/gpl-3.0.de.html}
\newcommand{\LICENSE}{https://www.gnu.org/licenses/gpl-3.0.de.html}

% Jede Überschrift 1 auf neuer Seite
\let\stdsection\section
\renewcommand\section{\clearpage\stdsection}

% Multiple Authors
\usepackage{authblk}

% Layout / Seitenränder
\usepackage{geometry}

% Inhaltsverzeichnis
\usepackage{makeidx} 
\makeindex

\usepackage{url}
\usepackage[pdfborder={0 0 0}]{hyperref}
\usepackage[all]{hypcap}
\usepackage{hyperxmp} % for license metadata

% Glossar und Abkürzungsverzeichnis
\usepackage[acronym,toc,nopostdot]{glossaries}
\glossarystyle{altlisthypergroup}
\usepackage{xparse}
\DeclareDocumentCommand{\newdualentry}{ O{} O{} m m m m } {
	\newglossaryentry{gls-#3}{name={#5},text={#5\glsadd{#3}},
		description={#6},#1
	}
	\makeglossaries
	\newacronym[see={[Siehe:]{gls-#3}},#2]{#3}{#4}{#5\glsadd{gls-#3}}
}
\makeglossaries

% Mathematik
\usepackage{amsmath}
\usepackage{amssymb}
\usepackage{amsfonts}
\usepackage{enumitem}

% Images
\usepackage{graphicx}
\graphicspath{{images/}} % default paths

% Boxes
\usepackage{fancybox}

%Tables
\usepackage{tabu}
\usepackage{booktabs} % toprule, midrule, bottomrule
\usepackage{array} % for matrix tables

% Multi Columns
\usepackage{multicol}

% Header and footer
\usepackage{scrlayer-scrpage}
\setkomafont{pagehead}{\normalfont}
\setkomafont{pagefoot}{\normalfont}
\automark*{section}
\clearpairofpagestyles
\ihead{\headmark}
\ohead{\TITLE}
\cfoot{\pagemark}

% Pseudocode
\usepackage{algorithm}
\usepackage{algorithmic}

% Code Listings
\usepackage{listings}
\usepackage{color}
\usepackage{beramono}

\definecolor{DarkPurple}{rgb}{0.4, 0.1, 0.4}
\definecolor{DarkCyan}{rgb}{0.0, 0.5, 0.4}
\definecolor{LightLime}{rgb}{0.3, 0.5, 0.4}
\definecolor{Blue}{rgb}{0.0, 0.0, 1.0}

\lstdefinestyle{eclipse-style}{
	language=Java,  
	columns=flexible,
	showstringspaces=false,     
	basicstyle=\footnotesize\ttfamily, 
	keywordstyle=\bfseries\color{DarkPurple},
	commentstyle=\color{LightLime},
	stringstyle=\color{Blue}, 
	escapeinside={£}{£}, % latex scope within code      
	morekeywords={length},
	numbers=left,
	numberstyle=\tiny\color{black},
	frame=single,
}
\lstset{style=eclipse-style}


% Theorems \begin{mytheo}{title}{label}
\usepackage{tcolorbox}
\tcbuselibrary{theorems}
\newtcbtheorem[number within=section]{definiton}{Definition}%
{fonttitle=\bfseries}{def}
\newtcbtheorem[number within=section]{remember}{Merke}%
{fonttitle=\bfseries}{rem}

% Dokumentinformationen
\newcommand{\SUBJECT}{Dokumentation}
\newcommand{\TITLE}{Sunlight Alarm}
\newcommand{\SECONDAUTHOR}{Fabian Hauser}

% pdf metadata
\hypersetup{
	pdfauthor={\AUTHOR},
	pdftitle={\SUBJECT \TITLE},
	pdfcopyright={\LICENSE},
	pdflicenseurl={\LICENSEURL}
}

\begin{document}
	
% Front page
\title{\TITLE}
\subject{\SUBJECT}
\author{\SECONDAUTHOR}
\author{\AUTHOR}
\affil{\INSTITUTE}
\date{\today}
\maketitle

\vfill

% Licence
\paragraph{Lizenz} \hfill \\
\LICENSE

% Table of contents
\tableofcontents


% Glossar and acronyms (if included \loadglsentries{glossar})
\printglossary[type=\acronymtype]
\printglossary
\glsaddall


\section{Vorbereitung}

\subsection{Hardware}
Was du für deinen persönlichen Sunlight Alarm benötigst:
\begin{itemize}
	\item Raspberry Pi Zero mit 
	\begin{itemize}
		\item Raspbery Pi WiFi dongle
		\item micro-B USB Ladekabel
		\item microSD Card
		\item Dupont cable (auf beiden Seiten female) für die Verbindung zwischen Wireless Sender und Raspberry Pi Zero
	\end{itemize}
	\item nRF24L01+ 2.4Ghz Wireless Sender \footnote{http://www.dx.com/de/p/nrf24l01-2-4ghz-wireless-transceiver-module-black-149483}
	\item Mi-Light GU10 Wireless Lamp
\end{itemize}

\subsection{Funktchip Verbinden}
Der Funktchip wird nun mit den Dupont Kabel mit dem Raspberry Pi Zero verbunden. Dazu ist der Pin-Belegungs-Plan unter http://pinout.xyz/ sehr nützlich.
\begin{table}[h]
\centering
\begin{tabu} to \linewidth {l c c}
	\toprule
	Leitung & Pim beim nRF24L01+ & Pin beim Zero \\
	GND (Ground / Erdung) & 1 & 20 \\
	3.3 Volt & 2 & 17 \\
	CE (chip enable bzw. slave select) & 3 & 22 \\
	CSN (chip select not) & 4 & 24 \\
	SCLK (serial clock) & 5 & 23 \\
	MOSI (Master Output Slave Input) & 6 & 19 \\
	MISO (Master Input Slave Output) & 7 & 21 \\
	\bottomrule
	\end{tabu}
	\caption{Dupont Verbindungen zwischen Rasperberry Pi Zero und nRF24L01+}
\end{table}

\subsection{Raspberry aufsetzen}
\begin{enumerate}
	\item SD Karte mit dem Notebook verbinden
	\item 
	\item sudo apt-get update
	\item sudo apt-get dist-upgrade
\end{enumerate}

\subsection{Wireless konfigurieren}
\begin{enumerate}
\item SD Karte an Laptop anschliessen
\item mount 
\item cd run/media/.../...
\item sudo vim /etc/wpa\_supplicant/wpa\_supplicant.conf
\begin{lstlisting}[caption=/etc/wpa\_supplicant/wpa\_supplicant.conf]
network={
 ssid="ssid"
 psk="pw"
 proto=RSN
 key_mgmt=WPA-PSK
 scan_ssid=1
}
\end{lstlisting}
\item umount /run/media/.../...
\end{enumerate}

\subsubsection{Verbinden}
\begin{enumerate}
	\item ssh pi@192.168.x.y
	\item Passwort eingeben
\end{enumerate}

\section{Software}
\begin{enumerate}
	\item git clone https://github.com/TMRh20/RF24
	\item cd RF24
	\item make
	\item sudo make install
\end{enumerate}

\subsection{}
%TODO update repo
\begin{enumerate}
	\item cd /opt
	\item git clone https://github.com/fabianhauser/milight-muell
	\item cd ...
	\item make
\end{enumerate}

\section{Befehle}
\begin{table}[h]
	\centering
	\begin{tabu} to \linewidth {l c}
		\toprule
		Hexwert & Taste \\
		0x & a \\
		\bottomrule
	\end{tabu}
	\caption{Openmilight Hex Befehlstabelle}
\end{table}




\appendix

% Code Listings
\lstlistoflistings

% List of figures
\listoffigures

% List of tables
\listoftables

% Bibliography
\bibliographystyle{plain} 
\bibliography{literatur}

\end{document}